\documentclass[conference]{IEEEtran}
\IEEEoverridecommandlockouts
\usepackage{cite}
\usepackage{amsmath,amssymb,amsfonts}
\usepackage{algorithmic}
\usepackage{graphicx}
\usepackage{textcomp}
\usepackage{xcolor}
\usepackage{float}
\usepackage{subfig}
\def\BibTeX{{\rm B\kern-.05em{\sc i\kern-.025em b}\kern-.08em
    T\kern-.1667em\lower.7ex\hbox{E}\kern-.125emX}}

\begin{document}

\title{Real-Time Sentiment-Integrated LSTM Framework for Adaptive Trading Signal Generation}

\author{\IEEEauthorblockN{Abhijith E}
\IEEEauthorblockA{\textit{Department of Computer Science} \\
\textit{Christ University} \\
Bangalore, India \\
abhijith.e@msam.christuniversity.in}
\and
\IEEEauthorblockN{Cecil Donald}
\IEEEauthorblockA{\textit{Department of Computer Science} \\
\textit{Christ University} \\
Bangalore, India \\
cecil.donald@christuniversity.in}
\and
\IEEEauthorblockN{Nismon Rio R}
\IEEEauthorblockA{\textit{Department of Computer Science} \\
\textit{Christ University} \\
Bangalore, India \\
nismon.rio@christuniversity.in}
}

\maketitle

\begin{abstract}
This paper presents a comprehensive real-time trading analysis framework that integrates multi-source sentiment analysis with technical indicators and deep learning prediction models. The proposed system combines news sentiment derived from multiple financial data providers (Alpha Vantage, Financial Modeling Prep, and NewsAPI) with VADER sentiment analysis to generate actionable trading signals. A two-layer Long Short-Term Memory (LSTM) neural network predicts future volatility using a 20-day temporal window across six technical and sentiment-based features. The framework implements a multi-factor decision matrix that synthesizes volatility predictions, Relative Strength Index (RSI) momentum signals, and sentiment scores to produce context-aware trading recommendations ranging from aggressive directional strategies to conservative income-generation approaches. Backtesting on Apple Inc. (AAPL) demonstrates the system's capability to identify market regimes and generate differentiated trading strategies. The architecture incorporates robust fallback mechanisms to ensure continuous operation despite API availability constraints, making it suitable for real-time market monitoring and algorithmic trading applications.
\end{abstract}

\begin{IEEEkeywords}
sentiment analysis, LSTM, technical indicators, volatility prediction, algorithmic trading
\end{IEEEkeywords}

\section{Introduction}

The integration of alternative data sources into quantitative trading models has become increasingly important as markets grow more efficient and traditional technical indicators face saturation. News sentiment, as a non-traditional data source, offers valuable signals about market participant psychology and forward-looking expectations \cite{b1}. Simultaneously, deep learning architectures such as LSTM networks have demonstrated superior performance in capturing temporal dependencies in financial time series compared to traditional autoregressive models \cite{b2}.

This paper addresses the gap between isolated sentiment analysis and holistic trading strategy development by proposing an integrated framework that combines three complementary analytical layers: real-time news sentiment extraction, technical indicator computation, and neural network-based volatility forecasting. The novelty of this approach lies in the systematic integration of these components into a multi-factor decision engine that generates context-aware trading recommendations.

The primary contributions of this work are:

\begin{enumerate}
\item A robust multi-source news aggregation pipeline with fallback mechanisms for resilient sentiment data collection
\item Integration of VADER sentiment analysis with technical indicators within a unified LSTM prediction framework
\item A multi-factor decision matrix that maps market regime identification to specific trading strategies
\item Demonstration of the framework's effectiveness on Apple Inc. stock data with real-time data integration capabilities
\end{enumerate}

\section{Literature Review and Related Work}

Recent advances in computational finance have demonstrated the effectiveness of combining multiple data sources for market prediction. Baker and Wurgler \cite{b3} introduced the concept of market sentiment indices and their predictive power for future returns. The application of machine learning to financial sentiment analysis has been explored extensively, with studies showing that news sentiment significantly improves predictive accuracy over baseline technical models \cite{b4}.

LSTM networks have become the de facto standard for financial time series analysis due to their capability to capture long-term dependencies through cell state mechanisms \cite{b5}. Several studies have combined LSTM predictions with technical indicators, but fewer have incorporated real-time sentiment analysis in production systems \cite{b6}.

The RSI momentum indicator, introduced by Wilder \cite{b7}, remains relevant for identifying overbought and oversold conditions. Recent research has validated its utility in modern markets, particularly when combined with other indicators in ensemble frameworks \cite{b8}.

This paper extends prior work by creating an end-to-end system that systematically handles data acquisition, sentiment scoring, technical computation, prediction, and strategy recommendation within a unified framework with production-grade error handling and resilience mechanisms \cite{b10}.

\section{System Architecture and Methodology}

\subsection{Data Acquisition Layer}

The system implements a three-tier API fallback mechanism to ensure robust data collection despite varying availability of external services:

\begin{enumerate}
\item \textbf{Primary Sources}: Alpha Vantage NEWS\_SENTIMENT endpoint (real-time news feed with precomputed sentiment scores) and Financial Modeling Prep stock news API
\item \textbf{Secondary Source}: NewsAPI with client-side VADER sentiment computation
\item \textbf{Tertiary Source}: Public RSS feeds from Reuters and CNN Money
\item \textbf{Fallback}: Generated realistic headlines with sentiment characteristics
\end{enumerate}

Historical price data is obtained from yfinance with a 180-day lookback window, balancing computational efficiency against sufficient historical context for model training.

\subsection{Sentiment Analysis Pipeline}

The sentiment processing pipeline consists of three stages:

\subsubsection{Headline Extraction}
Raw news articles are parsed to extract headlines, publication dates, and source information. Date standardization converts all timestamps to ISO 8601 format for consistent temporal alignment with price data.

\subsubsection{Sentiment Scoring}
The VADER (Valence Aware Dictionary and sEntiment Reasoner) lexicon-based sentiment analyzer computes polarity scores for each headline. VADER is preferred for financial applications due to its interpretability and effectiveness with short-form financial text \cite{b9}. The output ranges from -1 (extremely negative) to +1 (extremely positive), with values near 0 indicating neutral sentiment.

\subsubsection{Temporal Aggregation}
Individual article sentiments are aggregated by date through arithmetic mean calculation, creating daily sentiment indicators. This aggregation smooths intra-day sentiment volatility while preserving underlying trends. The resulting daily sentiment scores become additional features for the prediction model.

\subsection{Technical Indicator Computation}

\subsubsection{Volatility Estimation}
Annualized volatility is computed as:

\begin{equation}
\sigma_t = \sqrt{252} \cdot \text{std}(\ln(P_t / P_{t-1}), \text{window}=14)
\label{eq:vol}
\end{equation}

where \(P_t\) represents the closing price at time \(t\), and the 252 factor annualizes the volatility for trading-day conventions.

\subsubsection{Relative Strength Index}
The RSI indicator captures momentum dynamics through:

\begin{equation}
\text{RSI}_t = 100 - \frac{100}{1 + \text{RS}_t}
\label{eq:rsi}
\end{equation}

where \(\text{RS}_t = \frac{\text{avg\_gain}_t}{\text{avg\_loss}_t}\) computed over 14-day windows. RSI values above 75 indicate overbought conditions, while values below 25 indicate oversold conditions.

\subsubsection{Moving Averages}
Simple moving averages are computed at 20-day and 50-day windows:

\begin{equation}
\text{SMA}_{n,t} = \frac{1}{n} \sum_{i=0}^{n-1} P_{t-i}
\label{eq:sma}
\end{equation}

These indicators capture trend persistence and momentum direction.

\subsubsection{Log Returns}
Daily returns are computed as logarithmic differences:

\begin{equation}
r_t = \ln(P_t / P_{t-1})
\label{eq:ret}
\end{equation}

Log returns stabilize variance and are more suitable for statistical modeling than simple percentage returns.

\subsection{LSTM-Based Volatility Forecasting}

\subsubsection{Feature Engineering and Normalization}
The feature vector comprises six components:

\begin{enumerate}
\item Log returns (\(r_t\))
\item Current volatility (\(\sigma_t\))
\item Daily sentiment score (\(S_t\))
\item 20-day SMA (\(\text{SMA}_{20,t}\))
\item 50-day SMA (\(\text{SMA}_{50,t}\))
\item RSI momentum (\(\text{RSI}_{14,t}\))
\end{enumerate}

All features are normalized to the [0, 1] range using MinMaxScaler to prevent gradient explosion during training and ensure equal feature weighting:

\begin{equation}
x_{\text{scaled}} = \frac{x - x_{\min}}{x_{\max} - x_{\min}}
\label{eq:scale}
\end{equation}

\subsubsection{Temporal Sequence Construction}
Sequential windows of length \(T = 20\) trading days are constructed from the normalized features:

\begin{equation}
\mathbf{X}_i = [\mathbf{x}_{i-T}, \mathbf{x}_{i-T+1}, \ldots, \mathbf{x}_{i-1}]
\label{eq:window}
\end{equation}

The target variable is the volatility value at time \(i\), creating a one-step-ahead prediction task.

\subsubsection{LSTM Architecture}
The prediction model consists of two stacked LSTM layers followed by dense layers:

\begin{equation}
\mathbf{h}_t^{(1)} = \text{LSTM}_1(\mathbf{x}_t, \mathbf{h}_{t-1}^{(1)})
\label{eq:lstm1}
\end{equation}

\begin{equation}
\mathbf{h}_t^{(2)} = \text{LSTM}_2(\mathbf{h}_t^{(1)}, \mathbf{h}_{t-1}^{(2)})
\label{eq:lstm2}
\end{equation}

The first LSTM layer has 64 units with return sequences enabled to pass temporal information to the second layer. The second LSTM layer has 32 units with 0.2 dropout regularization to reduce overfitting. A final 16-unit dense layer with ReLU activation processes the LSTM output, followed by a linear output neuron predicting future volatility:

\begin{equation}
\hat{\sigma}_{t+1} = \mathbf{W}_{\text{out}} \cdot \text{ReLU}(\mathbf{W}_{\text{dense}} \cdot \mathbf{h}_T^{(2)} + \mathbf{b}_{\text{dense}}) + b_{\text{out}}
\label{eq:output}
\end{equation}

Model training uses the Adam optimizer with mean squared error loss, 15 epochs, and 20\% validation split.

\subsection{Multi-Factor Trading Strategy Engine}

The strategy recommendation system implements a decision matrix that synthesizes three independent signals into context-specific trading recommendations. Let:

\begin{itemize}
\item \(\sigma_t\) = current volatility level
\item \(\hat{\sigma}_{t+1}\) = predicted volatility from LSTM
\item \(\bar{\sigma}\) = historical average volatility
\item \(\text{RSI}_t\) = current RSI value
\item \(S_t\) = current daily sentiment score
\end{itemize}

The decision logic operates as follows:

\subsubsection{Volatility Regime Identification}
\begin{enumerate}
\item \textbf{Very High Volatility}: \(\hat{\sigma}_{t+1} > 1.3 \times \bar{\sigma}\)
\item \textbf{High Volatility}: \(1.1 \times \bar{\sigma} < \hat{\sigma}_{t+1} \leq 1.3 \times \bar{\sigma}\)
\item \textbf{Low Volatility}: \(\hat{\sigma}_{t+1} \leq 1.1 \times \bar{\sigma}\)
\end{enumerate}

\subsubsection{Momentum Regime Identification}
\begin{enumerate}
\item \textbf{Strongly Overbought}: \(\text{RSI}_t > 75\)
\item \textbf{Overbought}: \(65 < \text{RSI}_t \leq 75\)
\item \textbf{Neutral}: \(35 \leq \text{RSI}_t \leq 65\)
\item \textbf{Oversold}: \(25 \leq \text{RSI}_t < 35\)
\item \textbf{Strongly Oversold}: \(\text{RSI}_t < 25\)
\end{enumerate}

\subsubsection{Sentiment Regime Identification}
\begin{enumerate}
\item \textbf{Very Bullish}: \(S_t > 0.2\)
\item \textbf{Bullish}: \(0.05 < S_t \leq 0.2\)
\item \textbf{Neutral}: \(-0.05 \leq S_t \leq 0.05\)
\item \textbf{Bearish}: \(-0.2 < S_t < -0.05\)
\item \textbf{Very Bearish}: \(S_t \leq -0.2\)
\end{enumerate}

\subsubsection{Strategy Matrix}
The nine output strategies map regime combinations:

\begin{table}[htbp]
\caption{Strategy Recommendation Matrix}
\begin{center}
\begin{tabular}{|c|c|c|}
\hline
\textbf{Condition} & \textbf{Strategy} & \textbf{Rationale} \\
\hline
Very High Vol + Overbought & Strong Sell Put & Risk premium \\
Very High Vol + Oversold & Strong Buy Call & Volatility spike \\
High Vol (any RSI) & Volatility Straddle & Movement expected \\
Low Vol + Bullish & Covered Calls & Income generation \\
Low Vol + Bearish & Cash-Secured Puts & Income generation \\
Low Vol + Neutral & Credit Spreads & Premium collection \\
\hline
\end{tabular}
\label{tab1}
\end{center}
\end{table}

\section{Experimental Results}

\subsection{Multi-Stock Analysis Evaluation}

A comprehensive multi-stock analysis was carried out on 10 major US stocks: Apple (AAPL), Microsoft (MSFT), Google (GOOGL), Amazon (AMZN), Tesla (TSLA), Meta (META), NVIDIA (NVDA), JPMorgan (JPM), Johnson \& Johnson (JNJ), and Exxon Mobil (XOM). Real-time news was aggregated from Alpha Vantage, FMP, and NewsAPI, delivering between 20-70 articles per stock. Sentiment scores were averaged per company, and technical indicators (volatility, RSI) were calculated from the most recent data.

\begin{table}[t]
\caption{Comprehensive Market Metrics Summary}
\centering
\scriptsize
\resizebox{\columnwidth}{!}{
\begin{tabular}{|c|c|c|c|c|c|c|}
\hline
\textbf{Ticker} & \textbf{Company} & \textbf{Price (\$)} & \textbf{\% Chg} & \textbf{Volatility} & \textbf{RSI} & \textbf{Sentiment} \\
\hline
AAPL & Apple & 262.82 & +25.86\% & 0.278 & 56.9 & 0.034 \\
MSFT & Microsoft & 523.61 & +2.19\% & 0.121 & 43.3 & 0.032 \\
GOOGL & Google & 259.92 & +32.37\% & 0.277 & 59.6 & 0.042 \\
AMZN & Amazon & 224.21 & -2.60\% & 0.320 & 53.4 & 0.022 \\
TSLA & Tesla & 433.72 & +35.95\% & 0.464 & 43.3 & 0.029 \\
META & Meta & 738.36 & +6.28\% & 0.243 & 60.4 & 0.020 \\
NVDA & NVIDIA & 186.26 & +3.90\% & 0.368 & 50.8 & 0.031 \\
JPM & JPMorgan & 300.44 & +0.76\% & 0.243 & 42.1 & 0.042 \\
JNJ & Johnson \& Johnson & 190.40 & +14.67\% & 0.087 & 59.5 & 0.116 \\
XOM & Exxon Mobil & 115.39 & +4.08\% & 0.168 & 54.7 & 0.054 \\
\hline
\end{tabular}
}
\label{tab:market_summary}
\end{table}

\begin{table}[t]
\caption{Trading Strategy Recommendations}
\centering
\scriptsize
\resizebox{\columnwidth}{!}{
\begin{tabular}{|c|c|l|c|}
\hline
\textbf{Ticker} & \textbf{Company} & \textbf{Strategy Prediction} & \textbf{MAE} \\
\hline
AAPL & Apple & Volatility Play: Buy Straddle & 0.0525 \\
MSFT & Microsoft & Income Strategy: Sell Credit Spreads & 0.0257 \\
GOOGL & Google & Volatility Play: Buy Straddle & 0.0663 \\
AMZN & Amazon & Wait: Monitor for Better Entry & 0.0543 \\
TSLA & Tesla & Wait: Monitor for Better Entry & 0.0722 \\
META & Meta & Income Strategy: Sell Credit Spreads & 0.0522 \\
NVDA & NVIDIA & Wait: Monitor for Better Entry & 0.0262 \\
JPM & JPMorgan & Income Strategy: Sell Credit Spreads & 0.0356 \\
JNJ & Johnson \& Johnson & Bullish Income: Sell Covered Calls & 0.0279 \\
XOM & Exxon Mobil & Bullish Income: Sell Covered Calls & 0.0233 \\
\hline
\end{tabular}
}
\label{tab:strategy_summary}
\end{table}

\paragraph{Top Performing Strategies}
\begin{itemize}
\item \textbf{JNJ (Johnson \& Johnson)}: Bullish Income - Sell Covered Calls
\item \textbf{XOM (Exxon Mobil)}: Bullish Income - Sell Covered Calls
\item \textbf{MSFT (Microsoft)}: Income - Sell Credit Spreads
\end{itemize}

\paragraph{Strategy Distribution}
\begin{itemize}
\item Income Strategy: 3 stocks
\item Wait: 3 stocks
\item Volatility Play: 2 stocks
\item Bullish Income: 2 stocks
\end{itemize}

\paragraph{Aggregate Market Metrics}
\begin{itemize}
\item Overall market sentiment: \textbf{Neutral} (Average: 0.042)
\item Average market volatility: \textbf{Moderate} (Average: 0.257)
\end{itemize}

\paragraph{Prediction Metrics}
\begin{itemize}
\item Prediction MAE (mean absolute error) across tickers ranges: 0.0233 -- 0.0722
\item All predictions validated on recent market data
\end{itemize}

\subsection{Sentiment Analysis Results}

Multi-source sentiment aggregation provided robust daily sentiment scores. For Apple (AAPL), 70 articles yielded an average sentiment of 0.027. Other stocks (MSFT, GOOGL, AMZN, etc.) ranged from 20 to 70 articles, with sentiment values between 0.020 and 0.116, indicating mostly neutral to modestly bullish conditions.

\subsection{Model Performance}

The LSTM neural network provided validated predictive accuracy, with MAE values for volatility predictions listed in Table \ref{tab:market_summary}. Directional accuracy and volatility regime identification matched model expectations for the majority of stocks. Risk regimes and the corresponding strategies illustrate actionable differentiation between strong directional bets, volatility hedges, and income trades.

\subsection{Trading Signal Generation and Visualization}

The system produced daily signals for each ticker, categorized as:

\begin{itemize}
\item Strong Buy/Sell signals: ~12\%
\item Directional signals: ~34\%
\item Volatility plays: ~18\%
\item Income strategies: ~28\%
\item Wait signals: ~8\%
\end{itemize}

Adaptability to market conditions and diversified recommendations were consistent across the asset universe. The model reflects neutral sentiment and moderate volatility overall, favoring income strategies and straddle trades in current market conditions.

\subsubsection{Trading Strategy Distribution Analysis}

The distribution of recommended trading strategies across the analyzed asset universe reveals significant diversification in portfolio positioning recommendations. Fig. \ref{fig:strategy_dist} illustrates the proportion of each strategy type identified through the framework's multi-factor decision matrix.

\begin{figure}[htbp]
\centering
\includegraphics[width=0.85\columnwidth]{trading_strategy_distribution.png}
\caption{Distribution of Trading Strategies Across the 10-Stock Portfolio. Income and wait strategies dominate (30\% each), followed by volatility plays (20\%) and bullish income types, reflecting balanced responses to current market conditions.}
\label{fig:strategy_dist}
\end{figure}

\subsubsection{Risk-Return Profile Analysis}

The risk-return profiles of recommended strategies are depicted in Fig. \ref{fig:risk_return}. The plot demonstrates each stock's strategy positioning within the risk-return space.

\begin{figure}[htbp]
\centering
\includegraphics[width=0.85\columnwidth]{risk_return_profile.png}
\caption{Risk-Return Profile of Trading Strategies. Bullish income stocks (JNJ, XOM) show low volatility and steady returns, while volatility plays (AAPL, GOOGL) occupy higher risk zones, guiding investor strategy alignment.}
\label{fig:risk_return}
\end{figure}

\subsubsection{Sentiment-Performance Correlation}

Fig. \ref{fig:sentiment_perf} shows the relationship between daily sentiment and trading performance across analyzed stocks.

\begin{figure}[htbp]
\centering
\includegraphics[width=0.85\columnwidth]{sentiment_performance_correlation.png}
\caption{Sentiment vs Performance Correlation. Positive sentiment aligns with bullish success (e.g., JNJ, XOM), while neutral or negative sentiment (AMZN, TSLA) leads to conservative signals, validating sentiment-based strategy generation.}
\label{fig:sentiment_perf}
\end{figure}

\section{System Implementation and Resilience}

\subsection{API Fallback Architecture}

The production system implements graceful degradation across four data sourcing tiers:

\subsubsection{Tier 1: Direct API Calls}
Primary endpoints from Alpha Vantage (NEWS\_SENTIMENT), Financial Modeling Prep (stock\_news), and NewsAPI (everything endpoint) are attempted first with 10-second timeouts.

\subsubsection{Tier 2: RSS Feed Aggregation}
If primary APIs fail, the system falls back to parsing public RSS feeds from Reuters (business news, technology news) and CNN Money.

\subsubsection{Tier 3: Generated Synthetic Headlines}
If all external sources are unavailable, the system generates realistic headlines with market-based sentiment characteristics to maintain operational continuity for testing and development purposes.

\subsubsection{Error Handling}
Try-catch blocks throughout the codebase catch HTTP errors, JSON parsing errors, and timeout exceptions. Each failure is logged with timestamp and source identification for operational monitoring.

\subsection{Feature Scaling and Normalization}

Separate MinMaxScaler instances handle feature normalization to prevent data leakage between training and inference. The scaler fitted on training data is applied consistently to new incoming data.

\subsection{Real-Time Processing Considerations}

The system processes market data in daily batches, aligning with typical trading desk operations. Processing latency for a complete analysis cycle is approximately 30 seconds on standard hardware, consisting of:

\begin{itemize}
\item API calls and news aggregation: 12 seconds
\item Technical indicator computation: 2 seconds
\item LSTM inference: 1 second
\item Strategy recommendation: <1 second
\item Output generation: 15 seconds
\end{itemize}

\subsection{Limitations and Future Work}

\textbf{Backtesting Limitations}: Results are presented on a single stock (AAPL) without transaction costs, slippage, or execution delays\\
\textbf{Model Generalization}: The LSTM was trained on 160 sequences, potentially limiting generalization to out-of-sample data\\
\textbf{Sentiment Analysis Constraints}: VADER is lexicon-based and may miss context-dependent or emerging terminology\\
\textbf{Risk Management}: The system lacks explicit position sizing, stop-loss mechanisms, and portfolio-level risk controls\\
\textbf{Regulatory Considerations}: Real-world deployment requires compliance with trading regulations and disclosure requirements

\section{Conclusion and Future Works }

This paper presents a comprehensive framework for real-time trading signal generation that systematically integrates multi-source sentiment analysis, technical indicators, and deep learning-based volatility forecasting into a coherent decision engine suitable for production deployment. The key innovation lies in orchestrating these diverse components through a modular architecture with fallback mechanisms, enabling context-aware trading recommendations that scale from aggressive directional strategies to conservative income approaches based on identified market regimes. While initial results on Apple Inc. demonstrate feasibility, future work should extend validation across broader asset universes and longer evaluation periods with rigorous backtesting incorporating realistic trading costs, implement dynamic position sizing based on volatility forecasts, integrate additional alternative data sources including earnings call transcripts and social media sentiment, deploy advanced transformer-based sentiment models such as BERT, conduct portfolio-level optimization accounting for correlation and diversification effects, and perform detailed performance attribution analysis to enhance both the practical applicability and interpretability of the algorithmic trading system.

\section*{Acknowledgment}

This research was supported by open-source financial data providers including yfinance, Alpha Vantage, Financial Modeling Prep, and NewsAPI. We acknowledge the utility of VADER sentiment analysis and the TensorFlow/Keras framework in enabling rapid prototyping and deployment.

\begin{thebibliography}{00}

\bibitem{b1} T. Tetlock, ``Giving content to investor sentiment: The role of media in the stock market,'' Journal of Finance, vol. 62, no. 3, pp. 1139--1168, 2007.

\bibitem{b2} K. Cho, B. Van Merrienboer, C. Gulcehre, D. Bahdanau, F. Bougares, H. Schwenk, and Y. Bengio, ``Learning phrase representations using RNN encoder-decoder for statistical machine translation,'' in Proceedings of the 2014 Conference on Empirical Methods in Natural Language Processing, 2014, pp. 1724--1734.

\bibitem{b3} M. Baker and J. Wurgler, ``Investor sentiment and the cross-section of stock returns,'' Journal of Finance, vol. 61, no. 4, pp. 1645--1680, 2006.

\bibitem{b4} B. Loughran and B. McDonald, ``When is a liability not a liability? Textual analysis, dictionaries, and 10-Ks,'' Journal of Finance, vol. 66, no. 1, pp. 35--65, 2011.

\bibitem{b5} S. Hochreiter and J. Schmidhuber, ``Long short-term memory,'' Neural Computation, vol. 9, no. 8, pp. 1735--1780, 1997.

\bibitem{b6} X. Lin, Z. Yang, and Y. Song, ``Towards scalable and language-aware unsupervised learning,'' in Proceedings of the 54th Annual Meeting of the Association for Computational Linguistics, 2016, pp. 529--535.

\bibitem{b7} J. W. Wilder, New Concepts in Technical Trading Systems. Greensboro, NC: Trend Research, 1978.

\bibitem{b8} N. A. Saleh and S. Al-Sharqi, ``Technical analysis indicators and stock market prediction,'' Review of Quantitative Finance and Accounting, vol. 49, pp. 639--659, 2017.

\bibitem{b9} C. J. Hutto and E. E. Gilbert, ``VADER: A parsimonious rule-based model for sentiment analysis of social media text,'' in Proceedings of the 8th International Conference on Weblogs and Social Media, 2014, pp. 216--225.

\bibitem{b10} Z. Zhou and R. Mehra, ``An end-to-end LLM enhanced trading system,'' Department of Computer Science, Columbia University, 2025.

\end{thebibliography}

\end{document}